\section{Conclusions}\label{sec:con}


This paper presents the $\avats$ framework, which achieves high visual quality and low visualization latency for large-scale trajectory datasets. $\avats$ provides guaranteed visual quality in trajectory sampling by formulating a quality optimal sampling problem and developing effective solutions including $\vats$ and $\vats$. Low visualization latency is achieved with the $\invQ$-tree index, which allows to use the sampling results computed offline. Experiment results show that $\avats$ consistently provides high quality visualization in different cases and its visualization time is orders of magnitude shorter than full visualization.
%We plan to extend $\avats$ to support the specific trajectory features such as mulit-class characteristics in the future. 

%a novel sampling technique, $\avats$, that guarantees the visual quality of line-based trajectory visualization and alleviates the visual clutter problem. The effectiveness and efficiency of the proposed method are validated with real world visual analysis tasks and quantitatively performance measurements. Possible future directions include (i) improving visual quality by sampling trajectory segments instead of complete trajectories and (ii) developing advanced color encoding schemes to better describe the spatial distribution of the trajectories.
%extending our approaches to support the specific trajectory features such as mulit-class characteristics.

%we focus on the sampling approach of trajectory segments other than the whole trajectories to achieve higher visual fidelity.
%We will also develop different color encoding schema to present the spatial distribution of trajectories more precisely. Currently, the color of one trajectory keep the same, thus the color of the long trajectories may mislead the users because they may pass through many regions with different traffic crowdedness.
%We also consider to extend our approach to support the specific trajectory features such as mulit-class characteristics. 