%%
%% This is file `sample-authordraft.tex',
%% generated with the docstrip utility.
%%
%% The original source files were:
%%
%% samples.dtx  (with options: `authordraft')
%% 
%% IMPORTANT NOTICE:
%% 
%% For the copyright see the source file.
%% 
%% Any modified versions of this file must be renamed
%% with new filenames distinct from sample-authordraft.tex.
%% 
%% For distribution of the original source see the terms
%% for copying and modification in the file samples.dtx.
%% 
%% This generated file may be distributed as long as the
%% original source files, as listed above, are part of the
%% same distribution. (The sources need not necessarily be
%% in the same archive or directory.)
%%
%% The first command in your LaTeX source must be the \documentclass command.
\documentclass[sigconf,authordraft,review]{acmart}
%% NOTE that a single column version may be required for 
%% submission and peer review. This can be done by changing
%% the \doucmentclass[...]{acmart} in this template to 
%% \documentclass[manuscript,screen,review]{acmart}
%% 
%% To ensure 100% compatibility, please check the white list of
%% approved LaTeX packages to be used with the Master Article Template at
%% https://www.acm.org/publications/taps/whitelist-of-latex-packages 
%% before creating your document. The white list page provides 
%% information on how to submit additional LaTeX packages for 
%% review and adoption.
%% Fonts used in the template cannot be substituted; margin 
%% adjustments are not allowed.
%%
%% \BibTeX command to typeset BibTeX logo in the docs
\AtBeginDocument{%
  \providecommand\BibTeX{{%
    \normalfont B\kern-0.5em{\scshape i\kern-0.25em b}\kern-0.8em\TeX}}}

%% Rights management information.  This information is sent to you
%% when you complete the rights form.  These commands have SAMPLE
%% values in them; it is your responsibility as an author to replace
%% the commands and values with those provided to you when you
%% complete the rights form.
\setcopyright{acmcopyright}
\copyrightyear{2018}
\acmYear{2018}
\acmDOI{10.1145/1122445.1122456}

%% These commands are for a PROCEEDINGS abstract or paper.
\acmConference[Woodstock '18]{Woodstock '18: ACM Symposium on Neural
  Gaze Detection}{June 03--05, 2018}{Woodstock, NY}
\acmBooktitle{Woodstock '18: ACM Symposium on Neural Gaze Detection,
  June 03--05, 2018, Woodstock, NY}
\acmPrice{15.00}
\acmISBN{978-1-4503-XXXX-X/18/06}


% \usepackage{cite}
% \usepackage{amsmath,amssymb,amsfonts, amsthm}
\usepackage{algorithm, algorithmicx}
\usepackage[noend]{algpseudocode}
\usepackage{graphicx}
\usepackage{textcomp}

% Author defined
% Defined by authors--------------------------------------------
\newtheorem{problem}{Problem}
\newtheorem{lemma}{Lemma}
\newtheorem{theorem}{Theorem}
\newcommand{\Bo}[1]{{\color{red} Bo: #1}}
% \newcommand{\QM}[1]{{\color{blue} QM: #1}}
\newcommand{\QM}[1]{{\color{blue}{#1}}}

\newcommand{\D}{\mathcal{T}}
\newcommand{\V}{\mathsf{V}}
\newcommand{\oR}{\mathcal{R}}
\newcommand{\VV}{\mathtt{V}}
\newcommand{\QQ}{\mathtt{Q}}
\newcommand{\MU}{\mathsf{U}}
\newcommand{\vats}{\mathsf{VQGS}}
\newcommand{\vatss}{\mathsf{VQGS}^+}
\newcommand{\vatssce}{\mathsf{VQGS}^+\mathsf{CE}}
\newcommand{\rand}{\mathsf{Random}}
\newcommand{\full}{\mathsf{Full}}
\newcommand{\avats}{\mathsf{CheetahTraj}}
\newcommand{\cavats}{\mathsf{CheetahTraj}\mathsf{CE}}
\newcommand{\qtavats}{\mathsf{CheetahTraj}-\mathsf{QT}}
\newcommand{\sz}{\textsf{Shenzhen}}
\newcommand{\pt}{\textsf{Porto}}
\newcommand{\cd}{\textsf{Chengdu}}
\newcommand{\prob}{\textsf{QOSP}}
\newcommand{\query}{\mathcal{Q}}
\newcommand{\wpts}{\mathsf{WayPoint}}
\newcommand{\invQ}{\mathsf{InvQuad}}
\newcommand{\II}{\mathcal{I}}

\newcommand{\trim}{\vspace{-2mm}}

\newcommand{\baseline}{\mathsf{baseline}}

\newcommand{\stitle}[1]{\vspace*{0.05em}\noindent{\bf #1:\/}}
\newcommand{\sstitle}[1]{\vspace*{0.05em}\noindent{\bf #1\/.}}

\thispagestyle{plain}
\pagestyle{plain}

\newcommand{\localavats}{\mathsf{local+fix}}

\newcommand{\squishlist}{
	\begin{list}{$\bullet$}
		{ \setlength{\itemsep}{0pt}
			\setlength{\parsep}{3pt}
			\setlength{\topsep}{3pt}
			\setlength{\partopsep}{0pt}
			\setlength{\leftmargin}{1.2em}
			\setlength{\labelwidth}{1em}
			\setlength{\labelsep}{0.6em}
		}
	}
	\newcommand{\squishend}{
	\end{list}
}

%%
%% Submission ID.
%% Use this when submitting an article to a sponsored event. You'll
%% receive a unique submission ID from the organizers
%% of the event, and this ID should be used as the parameter to this command.
%%\acmSubmissionID{123-A56-BU3}

%%
%% The majority of ACM publications use numbered citations and
%% references.  The command \citestyle{authoryear} switches to the
%% "author year" style.
%%
%% If you are preparing content for an event
%% sponsored by ACM SIGGRAPH, you must use the "author year" style of
%% citations and references.
%% Uncommenting
%% the next command will enable that style.
%%\citestyle{acmauthoryear}

%%
%% end of the preamble, start of the body of the document source.
\begin{document}

%%
%% The "title" command has an optional parameter,
%% allowing the author to define a "short title" to be used in page headers.
\title{$\avats$: Quality and Efficiency in Large-scale Trajectory Data Visual Exploration}

%%
%% The "author" command and its associated commands are used to define
%% the authors and their affiliations.
%% Of note is the shared affiliation of the first two authors, and the
%% "authornote" and "authornotemark" commands
%% used to denote shared contribution to the research.

%%\author{Ben Trovato}
%%\authornote{Both authors contributed equally to this research.}
%%\email{trovato@corporation.com}
%%\orcid{1234-5678-9012}
%%\author{G.K.M. Tobin}
%%\authornotemark[1]
%%\email{webmaster@marysville-ohio.com}
%%\affiliation{%
%%  \institution{Institute for Clarity in Documentation}
%%  \streetaddress{P.O. Box 1212}
%%  \city{Dublin}
%%  \state{Ohio}
%%  \country{USA}
%%  \postcode{43017-6221}
%%}

%%\author{Lars Th{\o}rv{\"a}ld}
%%\affiliation{%
%%  \institution{The Th{\o}rv{\"a}ld Group}
%%  \streetaddress{1 Th{\o}rv{\"a}ld Circle}
%%  \city{Hekla}
%%  \country{Iceland}}
%%\email{larst@affiliation.org}


%%
%% By default, the full list of authors will be used in the page
%% headers. Often, this list is too long, and will overlap
%% other information printed in the page headers. This command allows
%% the author to define a more concise list
%% of authors' names for this purpose.
% \renewcommand{\shortauthors}{Trovato and Tobin, et al.}

%%
%% The abstract is a short summary of the work to be presented in the
%% article.
\begin{abstract}
Visualizing large-scale trajectory data is a core subroutine for many applications, e.g., traffic management, urban planning, and route recommendation.
However, naively visualizing all trajectories for a target region could result in long delay due to large data volume.
Ad-hoc sampling reduces visualization time but may harm visual quality, i.e., generating visualizations that look substantially different from the exact one.
In this paper, we propose the $\avats$ framework to provide high quality trajectory visualization with low latency.
To this end, we first define a natural pixel-based \textit{visual quality function} to measure the similarity between two visualizations and formulate a quality optimal trajectory sampling problem.
Then, we design the $\vats$ and $\vatss$ algorithms to solve the trajectory sampling problem, which not only provide guaranteed visual quality but also reduce visual clutter.
To generate quality guaranteed trajectory samples with high efficiency, we develop a quad-tree-based index ($\invQ$) that allows to use trajectory samples computed offline.
%Extensive experiments (i.e., case-, user-, and quantitative- studies) are conducted on 3 real-world trajectory datasets to verify visualization quality and efficiency of $\avats$. The results show that $\avats$ consistently provide high quality visualizations and its visualization time is orders of magnitude shorter than visualizing all trajectories.   	
Extensive experiments (i.e., case-, user-, and quantitative- studies) are conducted on 3 real-world trajectory datasets and the results show that $\avats$ consistently provide higher visualization quality and better efficiency than the baselines.  
\end{abstract}

%%
%% The code below is generated by the tool at http://dl.acm.org/ccs.cfm.
%% Please copy and paste the code instead of the example below.
%%
\begin{CCSXML}
<ccs2012>
 <concept>
  <concept_id>10010520.10010553.10010562</concept_id>
  <concept_desc>Computer systems organization~Embedded systems</concept_desc>
  <concept_significance>500</concept_significance>
 </concept>
 <concept>
  <concept_id>10010520.10010575.10010755</concept_id>
  <concept_desc>Computer systems organization~Redundancy</concept_desc>
  <concept_significance>300</concept_significance>
 </concept>
 <concept>
  <concept_id>10010520.10010553.10010554</concept_id>
  <concept_desc>Computer systems organization~Robotics</concept_desc>
  <concept_significance>100</concept_significance>
 </concept>
 <concept>
  <concept_id>10003033.10003083.10003095</concept_id>
  <concept_desc>Networks~Network reliability</concept_desc>
  <concept_significance>100</concept_significance>
 </concept>
</ccs2012>
\end{CCSXML}

\ccsdesc[500]{Computer systems organization~Embedded systems}
\ccsdesc[300]{Computer systems organization~Redundancy}
\ccsdesc{Computer systems organization~Robotics}
\ccsdesc[100]{Networks~Network reliability}

%%
%% Keywords. The author(s) should pick words that accurately describe
%% the work being presented. Separate the keywords with commas.
\keywords{	
Trajectory visualization;
Interactive data exploration; Sampling}

%% A "teaser" image appears between the author and affiliation
%% information and the body of the document, and typically spans the
%% page.
%% Teaser
 \begin{teaserfigure}
 	\centering
    \includegraphics[width=0.82\textwidth]{pictures/case_study_icde/case_study_overview.pdf}
    \vspace{-2mm}
    \caption{Effectiveness of $\avats$ for overview visualization on the \pt{} dataset.}
    \Description{}
    \label{fig:overview}
 \end{teaserfigure}

%%
%% This command processes the author and affiliation and title
%% information and builds the first part of the formatted document.
\maketitle

\input{sections/Introduction.tex}

\section{Background and Related work}\label{sec:rel}
%In this section, we survey previous work and focus on the most relevant pieces.
%Section~\ref{sec:trajvisana} and ~\ref{sec:interactive} summarize the related works in trajectory visual analysis and interactive data visualization for large dataset, respectively.

In this part, we survey related works on \textit{trajectory visualization methods} in Section~\ref{sec:trajvisana} and \textit{interactive data visualization for large datasets} in Section~\ref{sec:interactive}, respectively.

\subsection{Trajectory Visualization Methods}\label{sec:trajvisana}
A trajectory is a sequence of spatial locations (e.g., GPS positioning results) and trajectories are the most common representations of object movements. Existing trajectory visualization methods can be classified into three categories according to the form of visualization~\cite{chen2015survey}, i.e., \textit{point-based}, \textit{region-based}, and \textit{line-based}. We give a brief introduction to these methods and refer the interested readers to~\cite{chen2015survey} for more detailed discussions.


Point-based visualization plots the locations in the trajectories independently and captures the overall spatial distribution of the moving objects. Many density-based methods~\cite{liu2013vait,yang2016exploring,chae2014public,borruso2008network}, 
%~\cite{liu2013vait,yang2016exploring,chae2014public,xie2008kernel, borruso2008network}
e.g., kernel density estimation, are applied in point-based visualization to preserve the spatial distribution. Region-based visualization slices the entire region into sub-regions and visualizes the aggregated information in each sub-region~\cite{guo2009flow,von2015mobilitygraphs}.
%~\cite{guo2009flow,wood2010visualisation,von2015mobilitygraphs}
As region-based visualization focuses on aggregated statistics, it is most effective in capturing macro-patterns. In this work, we focus on line-based visualization, which uses polylines to connect the locations in each trajectory and shows the trace of object movements (see an example in Figure~\ref{fig:line}). As line-based visualization preserves the continuous movement information of objects~\cite{guo2011tripvista,hurter2009fromdady}, it is widely used in many visual analysis applications such as traffic management, urban planning, and route recommendation. However, line-based visualization is known to suffer from severe visual clutter, especially when the dataset is large. Several techniques have been proposed to alleviate visual clutter, such as clustering-based techniques~\cite{von2015mobilitygraphs}
%~\cite{ferreira2013vector, rinzivillo2008visually, von2015mobilitygraphs}
and advanced interaction techniques~\cite{ferreira2013visual}.
%~\cite{kruger2013trajectorylenses, ferreira2013visual}



\subsection{Interactive Visualization for Large Datasets}\label{sec:interactive}


Figure~\ref{fig:sys_framework} illustrates a general architecture of interactive visualization systems,
e.g., Spotfire\footnote{\url{https://www.tibco.com/products/tibco-spotfire}}, Tableau\footnote{\url{https://www.tableau.com/}}, ATLAS~\cite{chan2008maintaining}, Viate~\cite{yang2019vaite} and Marviq~\cite{dong2020marviq}.
There are typically three layers: user interface in the front-end layer, optimization techniques in the middle-layer, and database management system (usually cloud-based) in the back-end layer. The visualization community usually focuses on improving the effectiveness of data visualization at the front-end, e.g., designing novel visualization methods/toolkits such as D3\footnote{\url{https://d3js.org/}} to enable data analysts to effectively gain insights from data. 
The database community usually aims to improve query efficiency, e.g., devising big data systems such as Spark\footnote{\url{https://spark.apache.org/}} for efficient data processing at the back-end. 
With the popularization of location-acquisition devices, the scale of trajectory datasets can be extremely large. For example, the taxis in Shenzhen generate {$\sim$}9.3GB trajectory data per day. However, visualization generation has a long latency for large datasets due to heavy data processing/graphic rendering, which harms the responsiveness of interactive visualization. Therefore, both the visualization and database communities began to advance techniques in the middle-layer to reduce visualization latency for large datasets. We briefly elaborate these techniques as follows.


\begin{figure}
	\centering
	\includegraphics[width=0.36\textwidth]{pictures/framework/framework.pdf}
	\trim
	\caption{System architecture for interactive visualization.} \label{fig:sys_framework}
    \trim \trim
\end{figure}


\stitle{Aggregation-based techniques}
These works divide the {entire area} into basic units and visualize the aggregated information of the trajectories for each unit~\cite{wood2010visualisation,guo2009flow,von2015mobilitygraphs}. For more details on aggregation-based techniques, we refer the reader to~\cite{andrienko2008spatio,adrienko2010spatial}. Our problem and solutions are different from these works as we focus on visualizing the raw trajectories, instead of aggregated statistics.


\stitle{Sampling-based techniques} Sampling is widely used in both visualization and database communities ~\cite{battle2013dynamic,rapp2019void,chen2014visual,yu2020turbocharging,park2016visualization,qin2020making,DBLP:conf/sigmod/DingHCC016,DBLP:journals/pvldb/KimBPIMR15}. These works try to reduce the dataset to a subset with some special characteristics: such as blue noise property~\cite{rapp2019void}, multi-class property~\cite{chen2014visual} or maximize some user-defined quality~\cite{yu2020turbocharging}. The work most relevant to ours is~\cite{park2016visualization}, which is designed for scatter plots (a form of point-based visualization). It reduces the number of points in a plot while preserving the spatial distribution of the points in the original dataset. The techniques in~\cite{park2016visualization} cannot be applied to our trajectory visualization problem
as trajectory is more complex than individual scatter points (e.g., the order of GPS points is essential and the trajectories could have a large variance in length). Some works simplify a trajectory by sampling important points to reduce data size~\cite{zhang2018trajectory,2018arXiv180303550V} or alleviate visual clutter~\cite{borcan2012improving, 6851202}. These works are orthogonal to ours as we are sampling complete trajectories instead of points in a trajectory.

\stitle{Caching-based and other techniques}
Chan et al. propose ATLAS~\cite{chan2008maintaining}, which utilizes caching for efficient data communication between server and client.
ATLAS also exploits a powerful multi-core server to accelerate visual analysis tasks in both the middle-layer and back-end.
Piringer et al.~\cite{piringer2009multi} propose an architecture for interactive visual exploration,
which utilizes multi-core devices and avoids the common pitfalls of multi-threading to provide quick visual feedback.
Our work is orthogonal to these execution optimizations as we mainly focus on the algorithm perspective.

\sstitle{Novelties of our work} To the best of our knowledge, we are the first to formulate the quality optimal trajectory sampling problem to accelerate visualization on large scale datasets. We devise effective algorithms for this problem, which not only provide visual quality guarantee but also reduce the well-know visual clutter in trajectory visualization. Based on these algorithm, we design the $\avats$ framework with a tailored $\invQ$-tree index to produce high quality visualization for arbitrary target region with low latency.



\input{sections/ProblemFormulation.tex}

\input{sections/ProblemSolving.tex}

\section{The $\avats$ Framework}\label{sec:cheetahtraj}
Recall that our goal is to provide high quality trajectory visualization for any user selected region query with low latency.
In this section, we first introduce the motivation behind the $\avats$ framework, then present its two key procedures: \textit{index building}, and \textit{query processing}.

\stitle{Motivation of $\avats$}
Given a user selected region query $\query$, a naive visualization procedure with our sampling algorithms works as follows:
it first retrieves all trajectories (or trajectory segments) that are in this region (a.k.a, $\wpts$ query~\cite{kruger2013trajectorylenses}),
then it invokes $\vatss$ (or $\vats$) to obtain a set $\oR$ of sample trajectories, and finally the trajectories in $\oR$ are rendered to the canvas (e.g., displaying device) as the visualization result.
$\vatss$ has short \textit{visualization time} as it effectively reduces the number of processed locations by sampling. However, $\vatss$ has a long \textit{sampling time} (e.g., several seconds to tens of seconds) even with our performance optimization techniques.
Hence, the naive procedure can not achieve low latency for large-scale trajectory visualization. To tackle this problem, we propose the $\avats$ framework as illustrated in Figure~\ref{fig:framework}.  $\avats$ consists of three modules: (i) \textsf{index building}, (ii) \textsf{query processing}, and (iii) \textsf{result visualization}, we elaborate them as follows.


\begin{figure}
	\centering
	\includegraphics[width=0.47\textwidth]{pictures/cheetahtraj}
    \trim
    \caption{The $\avats$ framework}
    \label{fig:framework}
    \trim
\end{figure}


\subsection{Index Building}~\label{sec:index}
The key idea of $\avats$ is to conduct $\vatss$ sampling in the offline \emph{index building} phase such that the sampling results can be used directly for online visualization.
Specifically, we propose an inverted list augmented quad-tree index ($\invQ$) to handle arbitrary query region.

As shown in the example $\invQ$-tree index $\II$ at the bottom of Figure~\ref{fig:framework}, we exploit a quad-tree to recursively partition the entire area (spanned by the trajectory dataset) into smaller areas and manage each area with a tree node.
For each tree node, we run $\vatss$ using the trajectories (or trajectory segments) in its associated area as input to compute the \textit{visualization quality inverted lists} for this area. $\mathcal{L}_0$ and $\mathcal{L}_{64}$ in Figure~\ref{fig:framework} are two example visualization quality inverted lists, in which the subscripts are the values of $\delta$ for this list.
Specifically, we compute several inverted lists with different $\delta$ values\footnote{We set $\delta$ as 0 (i.e., $\vats$), 4, 8, 16, 32, 64. We need quality inverted lists with different $\delta$ for one area as the area may be covered by query regions of different sizes, and we use lists with larger $\delta$ for larger query region as discussed in Section~\ref{subsec:VQGS+}.} to support the efficient quality guaranteed result visualization at various zoom levels.
For each inverted list, (i) $\vatss$ terminates until the quality of the sample set is $100\%$, i.e., the visualization result of the sample set is the same as the full dataset;
(ii) the trajectory selected at each iteration of $\vatss$ is stored in the inverted list with its \textit{cumulative quality} in ascending order. Take inverted list $\mathcal{L}_0$ in Figure~\ref{fig:framework} for example, $t_{i4}$ is the trajectory selected at the $4$th iteration,
$t_{i4}$'s cumulative quality is $0.23\%$, which means that the quality achieved by $\{t_{i1}, t_{i2}, t_{i3}, t_{i4}\}$ as a whole is $0.23\%$. With the quality inverted list, searching a quality guaranteed sample set for a query region can be conducted efficiently via binary search.


\subsection{Query Processing}~\label{sec:query}
For a region visualization query $\query$ with quality threshold $\tau$, Algorithm~\ref{alg:query} summarizes the $\mathsf{Query}$ subroutine, which finds a quality guaranteed trajectory sample set $\oR$. The algorithm starts by invoking $\mathsf{Query}(\query, \tau, \II.root, \oR=\emptyset)$, i.e., from the root of $\invQ$-tree index $\II$ with an empty result set $\oR$. Then Algorithm~\ref{alg:query} transverses the tree nodes recursively. If node   $\mathcal{N}$ is a leaf node or its associated area is entirely contained in the query region, we retrieve a quality guaranteed trajectory set by calling subroutine $\mathsf{findRet}()$, which conducts binary search on the proper inverted list in $\mathcal{N}$ (Line~\ref{line:ret}). Otherwise, we call $\mathsf{Query}()$ on the four children nodes of $\mathcal{N}$ ( Line~\ref{line:valls}-\ref{line:valle}).


\begin{algorithm}
	\caption{$\mathsf{Query}$($\query$, $\tau$, $\invQ$ node $\mathcal{N}$, result $\oR$)}
	\label{alg:query}
	\begin{algorithmic}[1]
        \If{ $\mathcal{N}$ is leaf node or  $\mathcal{N}$ is entirely contained in $\query$}
            \State $\oR \leftarrow \oR \cup \mathsf{findRet}(\mathcal{N}, \tau)$ \label{line:ret}
        \ElsIf {$\query \cap \mathcal{N} \neq \emptyset$} \label{line:valls}
            \For { $i$ from $0$ to $3$}
                \State $\mathsf{tmpQ} \leftarrow \query \cap \mathcal{N}.child[i]$
                \State $\mathsf{Query}(\mathsf{tmpQ}, \tau, \mathcal{N}.child[i], \oR)$
            \EndFor \label{line:valle}
        \EndIf
	\end{algorithmic}
\end{algorithm}

Some trajectories in $\oR$, the result returned by $\mathsf{Query}()$ for region $\query$, may have segments outside $\mathcal{Q}$,
we conduct a way point query $\wpts(\query, \oR)$ to filter these segments before visualization.
%

\stitle{Correctness analysis}
We first show that $\avats$ meets the visualization quality requirement in Theorem~\ref{theorem:quality} as follows.

\begin{theorem}\label{theorem:quality}
If all selected nodes in the $\invQ$-tree index $\II$ are entirely contained in the query region $\query$,
then the result set $\oR$ returned by Algorithm~\ref{alg:query} satisfies that $\QQ(\oR) \ge \tau$.
\end{theorem}

\begin{proof}
Suppose query region $\query$ selects areas $\mathcal{A}_1,\!\mathcal{A}_2,\!\cdots,\!\mathcal{A}_K$,
these areas satisfy $\mathcal{A}_i \cap \mathcal{A}_j = \emptyset$ for $i\neq j$, and $\cup_{k=1}^{K}\mathcal{A}_k=\mathcal{Q}$.
For each area $\mathcal{A}_k$, denote the number of points marked in the ground truth visualization as $n_k$,
and the number of points marked by the trajectories in $\mathcal{R}$ as $m_k$,
we have $\frac{m_k}{n_k} \ge \tau$ as we use the visualization quality inverted index for trajectory selection.
Thus, for query region $\query$ with result set $\oR$, we have
$\QQ(\oR) = \frac{\sum_{k=1}^{K} m_k}{\sum_{k=1}^{K} n_k} \ge \tau.$
\end{proof}
In more general cases, we also select some areas that only intersect with the query region $\query$ and the sample set $\oR$ may not satisfy $\frac{m_k}{n_k}\ge \tau$ for these areas.
This does not significantly affect visualization quality for two reasons:
(i) these areas are the leaf nodes of the $\invQ$-tree index and thus reside on the border of the query region.
When exploring the map, human tends to move the region of interest to the screen center, where is more ``close'' to eyes~\cite{fitts_click}. %fitts,
(ii) the areas of the border regions are small w.r.t. the query region if the $\invQ$-tree has a sufficient height (i.e., the leaf nodes have a small area).



\input{sections/Evaluation.tex}

\section{Conclusions}\label{sec:con}


This paper presents the $\avats$ framework, which achieves high visual quality and low visualization latency for large-scale trajectory datasets. $\avats$ provides guaranteed visual quality in trajectory sampling by formulating a quality optimal sampling problem and developing effective solutions including $\vats$ and $\vats$. Low visualization latency is achieved with the $\invQ$-tree index, which allows to use the sampling results computed offline. Experiment results show that $\avats$ consistently provides high quality visualization in different cases and its visualization time is orders of magnitude shorter than full visualization.
%We plan to extend $\avats$ to support the specific trajectory features such as mulit-class characteristics in the future. 

%a novel sampling technique, $\avats$, that guarantees the visual quality of line-based trajectory visualization and alleviates the visual clutter problem. The effectiveness and efficiency of the proposed method are validated with real world visual analysis tasks and quantitatively performance measurements. Possible future directions include (i) improving visual quality by sampling trajectory segments instead of complete trajectories and (ii) developing advanced color encoding schemes to better describe the spatial distribution of the trajectories.
%extending our approaches to support the specific trajectory features such as mulit-class characteristics.

%we focus on the sampling approach of trajectory segments other than the whole trajectories to achieve higher visual fidelity.
%We will also develop different color encoding schema to present the spatial distribution of trajectories more precisely. Currently, the color of one trajectory keep the same, thus the color of the long trajectories may mislead the users because they may pass through many regions with different traffic crowdedness.
%We also consider to extend our approach to support the specific trajectory features such as mulit-class characteristics. 
%%
%% The acknowledgments section is defined using the "acks" environment
%% (and NOT an unnumbered section). This ensures the proper
%% identification of the section in the article metadata, and the
%% consistent spelling of the heading.
% \begin{acks}
% To Robert, for the bagels and explaining CMYK and color spaces.
% \end{acks}

%%
%% The next two lines define the bibliography style to be used, and
%% the bibliography file.
\bibliographystyle{ACM-Reference-Format}
\bibliography{ref}

%%
%% If your work has an appendix, this is the place to put it.
% \appendix
% \section{Research Methods}
% \subsection{Part One}

% Lorem ipsum dolor sit amet, consectetur adipiscing elit. Morbi
% malesuada, quam in pulvinar varius, metus nunc fermentum urna, id
% sollicitudin purus odio sit amet enim. Aliquam ullamcorper eu ipsum
% vel mollis. Curabitur quis dictum nisl. Phasellus vel semper risus, et
% lacinia dolor. Integer ultricies commodo sem nec semper.

\end{document}
\endinput
%%
%% End of file `sample-authordraft.tex'.
